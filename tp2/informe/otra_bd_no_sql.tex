\section{Otras base de datos NoSql.}


Elegimos bases de datos key-value, el caso particular que analizamos es redis.
Redis permite el uso de namespaces (tener varios "diccionarios", la terminología de redis para esto sería "multiple databases"), lo cual hace el diseño más prolijo y simple.

Usar map-reduce en redis no es algo built-in ni estandar, pero investigamos y es posible, por
    ejemplo, integrarlo con Hadoop.

\subsection{Parte 1}
\begin{description}
\item[1a] En redis hay un comando SCAN que permite iterar las claves.
        Con esto podemos iterar un diccionario de empleados, donde en cada
            value hay una lista de datos de cada cliente que atendió
            ( empleado -> [cliente] ).
        Entonces se puede iterar las claves una por una y quedarse con las que
            tienen algun cliente mayor de edad.
\item[1b]
        Usando SCAN se puede iterar las claves de un diccionario
        articulos -> [ventas]. Mediante codigo se buscan las claves que tengan
        |ventas| maximo.
\item[1c]
        Idem usando SCAN. Si tenemos un diccionario sector -> [empleados], es cuestion
            de iterar y mediante código buscar cuando |empleados|==3

\item[1d]
        Un diccionario empleado -> [sectores]. Idem 1c usando SCAN.

\item[1e]
        Diccionario cliente  -> [compras]. Idem 1c, pero ordenando mediante |compras|.

\item[1f]
        Esta consulta ya es un poco más compleja, hay varios enfoques posibles.
        Uno es mantener las cosas simples y directamente usar un diccionario
            edad -> cantidadDeCompras. Por ser tan específico, es un diccionario
            que sólo sirve para esta consulta, con lo cual estamos agregando
            un costo de mantenimiento extra a la DB sólo por una query.
        Otra opción es valerse de map reduce y usar alguno de los otros diccionarios,
            como el de 1a. Como ventaja, la consulta es simple y no hace falta crear un diccionario extra
            solo por esta consulta.
\end{description}


\subsection{Parte 2}
    Asumimos que se crea un id único para las disposiciones, podría ser un hash de sus datos o
        la combinación (numeroBoja, paginaInicial, PaginaFinal) que asumimos que identifica
        univocamente a la disposicion. Entonces se tiene un diccionario id -> disposicion,
        sobre el cual podríamos correr todas las consultas por map reduce.

    Dichas resoluciones por map-reduce son prácticamente iguales a las versiones hechas en mongodb
        ya que la entrada de la función map es practicamente la misma.

%Parte 3:
    %TODO: primero habría que hacerlo en mongodb jaja.


