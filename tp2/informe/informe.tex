\documentclass[a4paper, 12pt, spanish]{article}

\usepackage[paper=a4paper, left=1.5cm, right=1.5cm, bottom=1.5cm, top=3.5cm]{geometry}
\usepackage[spanish, es-noshorthands]{babel}
\usepackage[utf8x]{inputenc}
\usepackage[none]{hyphenat}
\usepackage[colorlinks,citecolor=black,filecolor=black,linkcolor=black,    urlcolor=black]{hyperref}

% Simbolos matemáticos
\usepackage{amsthm}
\usepackage{amsmath}
\usepackage{amsfonts}
\usepackage{amssymb}
\usepackage{algorithm}
\usepackage[noend]{algpseudocode}
\usepackage{algorithmicx}
\usepackage{listings}
\lstset{
    breaklines=true,
    basicstyle=\normalfont\ttfamily,
    frame=single ,
    % basicstyle=\tiny,
    literate=%
        {á}{{\'a}}1
        {í}{{\'i}}1
        {é}{{\'e}}1
        {ú}{{\'u}}1
        {ó}{{\'o}}1
        {ñ}{{\~n}}1
        {Á}{{\'A}}1
        {Í}{{\'I}}1
        {É}{{\'E}}1
        {Ú}{{\'U}}1
        {Ó}{{\'O}}1
        {Ñ}{{\~N}}1
}

% Descoración y gráficos
\usepackage{caratulaV}
\usepackage{graphicx}
\usepackage{fancyhdr}
\usepackage{lastpage}
\usepackage{caption}
\usepackage{subcaption}
\usepackage{multirow}
\usepackage{alltt}
\usepackage{tikz}
\usepackage{color}
\usepackage{gnuplottex}
\usepackage{verbatim}
\usepackage{framed}
\usepackage[font=small,labelfont=bf]{caption}
\usepackage[normalem]{ulem} %para subrayar

% Bibliografía
\usepackage{natbib}

% Del enunciado
\usepackage{a4wide}
\usepackage{amsmath}
\usepackage{amsfonts}
\usepackage{graphicx}
%\usepackage[ruled,vlined]{algorithm2e}

\usepackage{bera}% optional: just to have a nice mono-spaced font
\usepackage{xcolor}

\newcommand{\kknn}{k}
\newcommand{\kpca}{\alpha}
\newcommand{\kkfold}{K}

% Acomodo fancyhdr.
\pagestyle{fancy}
\thispagestyle{fancy}
\addtolength{\headheight}{1pt}
\lhead{Bases de Datos}
\rhead{$2^{\mathrm{do}}$ cuatrimestre de 2015}
\cfoot{\thepage /\pageref*{LastPage}}
\renewcommand{\footrulewidth}{0.4pt}

\floatname{algorithm}{Pseudocódigo}
\algrenewcommand\algorithmicfunction{\textbf{Función}}
\algrenewcommand\algorithmicwhile{\textbf{mientras}}
\algrenewcommand\algorithmicfor{\textbf{para}}
\algrenewcommand\algorithmicforall{\textbf{para cada}}
\algrenewcommand\algorithmicdo{\textbf{hacer:}}
\algrenewcommand\algorithmicif{\textbf{si}}
\algrenewcommand\algorithmicthen{\textbf{entonces:}}
\algrenewcommand\algorithmicelse{\textbf{si no:}}
\algrenewcommand\algorithmicend{\textbf{fin}}
\algrenewcommand\algorithmicreturn{\textbf{devolver}}

\sloppy

\parskip=5pt % 10pt es el tama de fuente

% Pongo en 0 la distancia extra entre itemes.
\let\olditemize\itemize
\def\itemize{\olditemize\itemsep=0pt}

\usepackage{tikz}
%\usepackage{tikz-qtree}


\usetikzlibrary{arrows,backgrounds,calc}

\pgfdeclarelayer{background}
\pgfsetlayers{background,main}

\newcommand{\real}{\mathbb{R}}
\newcommand{\nat}{\mathbb{N}}

\newcommand{\revJ}[1]{{\color{red} #1}}

\newcommand{\convexpath}[2]{
[
    create hullnodes/.code={
        \global\edef\namelist{#1}
        \foreach [count=\counter] \nodename in \namelist {
            \global\edef\numberofnodes{\counter}
            \node at (\nodename) [draw=none,name=hullnode\counter] {};
        }
        \node at (hullnode\numberofnodes) [name=hullnode0,draw=none] {};
        \pgfmathtruncatemacro\lastnumber{\numberofnodes+1}
        \node at (hullnode1) [name=hullnode\lastnumber,draw=none] {};
    },
    create hullnodes
]
($(hullnode1)!#2!-90:(hullnode0)$)
\foreach [
    evaluate=\currentnode as \previousnode using \currentnode-1,
    evaluate=\currentnode as \nextnode using \currentnode+1
    ] \currentnode in {1,...,\numberofnodes} {
-- ($(hullnode\currentnode)!#2!-90:(hullnode\previousnode)$)
  let \p1 = ($(hullnode\currentnode)!#2!-90:(hullnode\previousnode) - (hullnode\currentnode)$),
    \n1 = {atan2(\x1,\y1)},
    \p2 = ($(hullnode\currentnode)!#2!90:(hullnode\nextnode) - (hullnode\currentnode)$),
    \n2 = {atan2(\x2,\y2)},
    \n{delta} = {-Mod(\n1-\n2,360)}
  in
    {arc [start angle=\n1, delta angle=\n{delta}, radius=#2]}
}
-- cycle
}

\newcommand{\todo}[1]{
\textbf{\color{red}{\underline{Nota:} #1}}
}

\newcommand\param[3]{\ensuremath{\mathbf{\textbf{#1}}\,#2\!:} \texttt{#3}}

\let\state\State
\let\while\While
\let\endwhile\EndWhile
\let\endif\EndIf
\let\elseif\ElsIf
\let\for\For
\let\endfor\EndFor
\let\function\Function
\let\endfunction\EndFunction


\newcommand{\degree}{\ensuremath{^\circ}}

\usepackage{caratula}
\materia{Bases de datos}
\submateria{Segundo Cuatrimestre de 2015}
\titulo{Trabajo Práctico II}
% \subtitulo{\emph{Reentrega}}
\fecha{\today}
\integrante{Ignacio Truffat}{837/10}{el\_truffa@hotmail.com}
\integrante{Gaston Rocca}{836/97}{gastonrocca@gmail.com}
\integrante{Agustín Godnic}{689/10}{agustingodnic@gmail.com}
\integrante{Matías Pizzagalli}{257/12}{matipizza@gmail.com}

\begin{document}
\maketitle

\newpage

\tableofcontents

\newpage
\section{Desnormalización.}
Pequeña intro al problema...

\subsection{Empleados que atendieron clientes mayores de edad.}

embebimos los clientes dentro de los empleados, con la fecha de atención, quedando nos asi:

\begin{lstlisting}
Empleado: {
  nroLegajo: int,
  nombre: string,
  clientes: [{DNI, Nombre, Edad, Fecha}],
  ...
}
\end{lstlisting}

Luego, para responder la consulta deseada hay que correr:

\begin{lstlisting}
db.empleados.aggregate(
  [
    { $unwind: "$clientes" },
    { $match: {"clientes.Edad": { $gt: 17 } }  },
    { $group: { _id: "$nombre", nombre:{$first:"$nombre" } } },
    { $project : { _id:0, nombre: 1 } }
  ]
)
\end{lstlisting}
\label{consulta-a}

\subsubsection{Ejemplo}

Para insertar registros en la base corremos:

\begin{lstlisting}
db.empleados.insert( { nroLegajo: 003, nombre: "Ernestino Juanes",
  clientes: [ {DNI: 40528343,  Nombre: "Raul Juan Lopez", Edad: 14,
  Fecha: "14/03/2015"} ] } )
db.empleados.insert( { nroLegajo: 002, nombre: "Juan Paez",
  clientes: [ {DNI: 30154820, Nombre: "Juana Perez", Edad: 23,
  Fecha: "20/04/2015"}, {DNI: 40528753, Nombre: "Raul Lopez", Edad: 15,
  Fecha: "23/04/2015"} ] } )
db.empleados.insert( { nroLegajo: 001, nombre: "Joaquina Paez",
  clientes: [ {DNI: 30154820,  Nombre: "Juan Perez", Edad: 25,
  Fecha: "03/04/2015"} ] } )
\end{lstlisting}

Luego, una vez insertados los registros deseados, corremos la consulta mencionada arriba y nos da:

\begin{lstlisting}
{ "nombre" : "Joaquina Paez" }
{ "nombre" : "Juan Paez" }
\end{lstlisting}

\subsection{Articulos mas vendidos.}
embebimos los DNIs de clientes q compraron dentro de los articulos y buscamos los maximos.

\begin{lstlisting}
Articulos: {
  CobBarras: int,
  nombre: string,
  compradores: [{DNI}]
}
\end{lstlisting}

Asumimos que hay un unico articulo mas vendido:
\begin{lstlisting}
db.articulos.aggregate(
  [
    { $unwind : "$Compradores"},
    {$group : { _id: "$CodBarras", CodBarras:{$first:"$CodBarras"},
      Nombre:{$first:"$Nombre"}, totalVendidos: {$sum: 1} } },
    {$sort: {totalVendidos: -1}},
    {$limit : 1},
    {$project : { _id:0,CodBarras: 1, totalVendidos: 1, Nombre: 1}}
  ]
)
\end{lstlisting}

\subsubsection{Ejemplo}

\begin{lstlisting}
db.articulos.insert( { CodBarras: 7231564345110546, Nombre: "1984",
  Compradores: [22222222] } )

db.articulos.insert( { CodBarras: 0231564105110546,
  Nombre: "El principito", Compradores: [333333333, 22222222] } )
\end{lstlisting}

\begin{lstlisting}
\end{lstlisting}

\subsection{Sectores donde trabaja exactamente 3 empleado.}

\subsubsection{Ejemplo}
\begin{lstlisting}
\end{lstlisting}

\subsection{Empleado que trabaja en más sectores..}

\subsubsection{Ejemplo}
\begin{lstlisting}
\end{lstlisting}

\subsection{Ranking de los clientes con mayor cantidad de compras.}

\subsubsection{Ejemplo}
\begin{lstlisting}
\end{lstlisting}


\subsection{Cantidad de compras realizadas por clientes de misma edad.}

\subsubsection{Ejemplo}
\begin{lstlisting}
\end{lstlisting}

\newpage
\section{Map-Reduce.}
Para la resolución de los Map-Reduce tuvimos que cargar diferentes archivos .json. Esto se realizo con las siguientes instrucciones:
\begin{lstlisting}
mongoimport --db DB --collection COLLECTION --file
disposiciones_201*.json --jsonArray

Por ejemplo:
mongoimport --db tp2 --collection disposiciones --file
disposiciones_2014.json --jsonArray
\end{lstlisting}

Luego implementamos las diferentes funciones de map y de reduce en un archivo aparte por claridad y comidad, llamado \texttt{code.js}.
Luego para cargar el código se puede hacer load("code.js") y podremos referenciar a las funciones implementadas en el archivo desde la consola de mongo.

\subsection{Disposiciones de tipo resolución en Abril de 2013.}
\textbf{Map:}
La idea de esta función es, si el documento dado es de tipo resolución y tiene fecha abril del 2013,
emitir("resolucion",cant=1). De esta forma estamos permitiendo que solo emitan los documentos que nos
interesa contar.

\begin{lstlisting}
var map1 = function(){
  var date = this["FechaBOJA"].split('/')
  if(this["Tipo"] == "Resoluciones" && date[1]==4 && date[2]==2013){
    emit(this["Tipo"],1)
  }
}
\end{lstlisting}

\textbf{Reduce:}
La idea del reduce es, sumar los valores que nos pasan, ya sean unos que vienen del map,
o valores parciales que vienen de otro reduce, lo que genera que este reduce sea combinable y se pueda
aplicar varias veces.

\begin{lstlisting}
var reduce1 = function(key,values){
  return Array.sum(values)
}
\end{lstlisting}

Una vez definadas las dos funciones podemos llamar a la función de map-reduce de la forma:
\begin{lstlisting}
db.disposiciones.mapReduce(map1,reduce1,{out: parte2a})
\end{lstlisting}

Lo que nos devuelve una collecion llamada \texttt{parte2a} con nuestra respuesta:

\begin{lstlisting}
> db.parte2a.find()
{ "_id" : "Resoluciones", "value" : 607 }
\end{lstlisting}

\subsection{Disposiciones de cada tipo.}
\textbf{Map:}
La idea de este map es, para cada documento, emitir su tipo y como valor un 1, para poder contar las ocurrencias
de cada tipo de documento.

\begin{lstlisting}
var map2 = function(){
	emit(this["Tipo"],1)
}
\end{lstlisting}

\textbf{Reduce:}
Esta función de reduce al igual que la anterior suma los valores que se le pasan y retorna la suma. De esta forma vamos a terminar
obteniendo la cantidad de ocurrencias de cada tipo posible de documento.
\begin{lstlisting}
var reduce2 = function(key,values){
	return Array.sum(values)
}
\end{lstlisting}

Una vez definadas las dos funciones podemos llamar a la función de map-reduce de la forma:
\begin{lstlisting}
db.disposiciones.mapReduce(map2,reduce2,{out: parte2b})
\end{lstlisting}

Lo que nos devuelve una collecion llamada \texttt{parte2b} con nuestra respuesta:

\begin{lstlisting}
db.parte2b.find()
{ "_id" : "", "value" : 3 }
{ "_id" : "Acuerdos", "value" : 2790 }
{ "_id" : "Acuerdos del Consejo de Gobierno", "value" : 160 }
{ "_id" : "Anuncios", "value" : 17226 }
{ "_id" : "Candidaturas", "value" : 1 }
{ "_id" : "Certificaciones", "value" : 72 }
{ "_id" : "Circular", "value" : 1 }
{ "_id" : "Conflictos Positivos", "value" : 6 }
{ "_id" : "Correcciones de Erratas", "value" : 59 }
{ "_id" : "Correcciones de Errores", "value" : 321 }
{ "_id" : "Correcciones de erratas", "value" : 8 }
{ "_id" : "Correcciones de errores", "value" : 44 }
{ "_id" : "Corrección de errores", "value" : 1 }
{ "_id" : "Correción de errores", "value" : 1 }
{ "_id" : "Cuestiones de Inconstitucionalidad", "value" : 1 }
{ "_id" : "Decretos", "value" : 828 }
{ "_id" : "Decretos Legislativos", "value" : 5 }
{ "_id" : "Decretos del Presidente", "value" : 13 }
{ "_id" : "Decretos-leyes", "value" : 15 }
{ "_id" : "Edictos", "value" : 3223 }
{ "_id" : "Instrucciones", "value" : 2 }
{ "_id" : "Leyes", "value" : 12 }
{ "_id" : "Notificaciones", "value" : 768 }
{ "_id" : "Orden", "value" : 4 }
{ "_id" : "Otros", "value" : 45 }
{ "_id" : "Reales Decretos", "value" : 5 }
{ "_id" : "Recursos de Inconstitucionalidad", "value" : 7 }
{ "_id" : "Requisitorias", "value" : 2 }
{ "_id" : "Resoluciones", "value" : 13956 }
{ "_id" : "Órdenes", "value" : 2061 }
{ "_id" : "Órdenes de Comision Delegada", "value" : 2 }
\end{lstlisting}

\subsection{Fecha mas citada.}

\textbf{Map:}
La idea de esta función es primero, parsear las fechas que encontramos en el documento exceptuando las que encontremos en la descripción (es decir la fechaBOJA y la fechaDisposicion) y hacer que sus formatos sean compatibles entre si.
Luego una vez obtenidas las fechas, emitimos una vez por cada fecha, tomando como clave la fecha con formato compatible y como valor un 1.
De esta forma podremos contar las ocurrencias de cada una de las fechas que aparecen citadas.

\begin{lstlisting}
var map3 = function(){
	var date = (this["FechaDisposicion"].split('T'))[0].
		split('-').reverse().join('/');
	emit(date,1);
	emit(this["FechaBOJA"],1);
}
\end{lstlisting}
\textbf{Reduce:}
La idea de este reduce al igual que los anteriores es simplemente sumar los valores que recibe.

\begin{lstlisting}
var reduce3 = function(key,values){
	return Array.sum(values)
}
\end{lstlisting}
Una vez definadas las dos funciones podemos llamar a la función de map-reduce de la forma:

\begin{lstlisting}
db.disposiciones.mapReduce(map3,reduce3,{out: parte2c})
\end{lstlisting}

Una vez ejecutado el mapReduce tendremos en la coleccion llamada \texttt{parte2c} todas las fechas citadas con sus respectivas cantidades de ocurrencias. Luego una vez tenemos esa informacion, lo unico que nos falta hacer es ordenar esta coleccion por cantidad de ocurrencias de manera descendente y quedarnos con el primer elemento, es decir:

\begin{lstlisting}
db.parte2c.find().sort({value : -1}).limit(1)
{ "_id" : "12/06/2012", "value" : 364 }
\end{lstlisting}

\subsection{Mayor cantidad de páginas por cada tipo.}

\textbf{Map:}
Lo que realiza la funcion map es, primero calcula la cantidad de paginas q ocupa el documento dado y luego emite su tipo como clave y la cantidad de paginas calculada como valor. Esto nos va a permitir calcular para cada tipo de documento, su máxima cantidad de paginas utilizada.

\begin{lstlisting}
var map4 = function(){
	var cantPags = this["PaginaFinal"] - this["PaginaInicial"] + 1
	emit(this["Tipo"],cantPags)
}
\end{lstlisting}
\textbf{Reduce:}
Esta función busca el maximo entre todos los valores que le pasan y lo devuelve, al emitir lo mismo que recibe este reduce tambien
es combinable y se puede aplicar mas de una vez.
\begin{lstlisting}
var reduce4 = function(key,values){
	var cantPagsMax = 0;
	for (var i=0; i < values.length; i++){
		if(values[i] > cantPagsMax){
			cantPagsMax = values[i];
		}
	}
	return(cantPagsMax);
}
\end{lstlisting}

Una vez definadas las dos funciones podemos llamar a la función de map-reduce de la forma:

\begin{lstlisting}
db.disposiciones.mapReduce(map4,reduce4,{out: parte2d})
\end{lstlisting}

Luego, en la coleccion \texttt{parte2d} encontraremos nuestra respuesta, es decir para cada tipo de documento la cantidad máxima de paginas utilizada por algun documento suyo.

\begin{lstlisting}
> db.parte2d.find()
{ "_id" : "", "value" : 16 }
{ "_id" : "Acuerdos", "value" : 12 }
{ "_id" : "Acuerdos del Consejo de Gobierno", "value" : 83 }
{ "_id" : "Anuncios", "value" : 393 }
{ "_id" : "Candidaturas", "value" : 68 }
{ "_id" : "Certificaciones", "value" : 47 }
{ "_id" : "Circular", "value" : 3 }
{ "_id" : "Conflictos Positivos", "value" : 1 }
{ "_id" : "Correcciones de Erratas", "value" : 174 }
{ "_id" : "Correcciones de Errores", "value" : 99 }
{ "_id" : "Correcciones de erratas", "value" : 10 }
{ "_id" : "Correcciones de errores", "value" : 5 }
{ "_id" : "Corrección de errores", "value" : 1 }
{ "_id" : "Correción de errores", "value" : 1 }
{ "_id" : "Cuestiones de Inconstitucionalidad", "value" : 1 }
{ "_id" : "Decretos", "value" : 492 }
{ "_id" : "Decretos Legislativos", "value" : 37 }
{ "_id" : "Decretos del Presidente", "value" : 3 }
{ "_id" : "Decretos-leyes", "value" : 139 }
{ "_id" : "Edictos", "value" : 48 }
{ "_id" : "Instrucciones", "value" : 3 }
{ "_id" : "Leyes", "value" : 194 }
{ "_id" : "Notificaciones", "value" : 11 }
{ "_id" : "Orden", "value" : 7 }
{ "_id" : "Otros", "value" : 65 }
{ "_id" : "Reales Decretos", "value" : 1 }
{ "_id" : "Recursos de Inconstitucionalidad", "value" : 1 }
{ "_id" : "Requisitorias", "value" : 1 }
{ "_id" : "Resoluciones", "value" : 454 }
{ "_id" : "Órdenes", "value" : 634 }
{ "_id" : "Órdenes de Comision Delegada", "value" : 2 }
\end{lstlisting}

\newpage
\section{Sharding.}
Levantamos cinco shards siguiendo las instrucciones del archivo tutorial\_sharding.txt.

Luego creamos un indice simple sobre el atributo codigo\_postal.

Luego importamos el código de insert\_data.js donde tenemos
funciones que nos permiten ingresar datos de a 20k y pedir las estadísticas.
Finalmente, nos guardamos las estadísticas en archivos .txt. Los mismos se encuentran en
la carpeta mediciones.

\begin{figure}[H]
\centering
\includegraphics[width=165mm]{../mediciones/sharding_simple.png}
\caption{Porcentaje de distribucion entre los shards usando un
indice simple en base al codigo postal.}
\end{figure}

\begin{figure}[H]
\centering
\includegraphics[width=165mm]{../mediciones/sharding_hashed.png}
\caption{Porcentaje de distribucion entre los shards usando un
indice hasheado en base al \_id}
\end{figure}



\newpage
\section{Otras base de datos NoSql.}


Base de datos key-value, usando el motor redis.
Redis permite el uso de namespaces (tener varios "diccionarios", la terminología de redis para esto sería "multiple databases"), lo cual hace el diseño más prolijo y simple.

Usar map-reduce en redis no es algo built-in ni estandar, pero investigamos y es posible, por
    ejemplo, integrarlo con Hadoop.

Parte1:
    1a:
        En redis hay un comando SCAN que permite iterar las claves.
        Con esto podemos iterar un diccionario de empleados, donde en cada
            value hay una lista de datos de cada cliente que atendió.
        Entonces se puede iterar las claves una por una y quedarse con las que
            tienen algun cliente mayor de edad.
    1b:
        Usando SCAN se puede iterar las claves de un diccionario
        articulos -> ventas. Mediante codigo se buscan las claves que tengan
        |ventas| maximo.

    1c:
        Idem usando SCAN. Si tenemos un diccionario sector -> [empleados], es cuestion
            de iterar y mediante código buscar cuando |empleados|==3

    1d:
        Un diccionario empleado -> [sectores]. Idem 1c usando SCAN.

    1e:
        Diccionario cliente  -> [compras]. Idem 1c, pero ordenando mediante |compras|.

    1f:
        Esta consulta ya es un poco más compleja, hay varios enfoques posibles.
        Uno es mantener las cosas simples y directamente usar un diccionario
            edad -> cantidadDeCompras. Por ser tan específico, es un diccionario
            que sólo sirve para esta consulta, con lo cual estamos agregando
            un costo de mantenimiento extra a la DB sólo por una query.
        Otra opción es valerse de map reduce y usar alguno de los otros diccionarios,
            como el de 1a. Como ventaja, la consulta es simple y no hace falta crear un diccionario extra
            solo por esta consulta.


Parte 2:
    Asumimos que se crea un id único para las disposiciones, podría ser un hash de sus datos o
        la combinación (numeroBoja, paginaInicial, PaginaFinal) que asumimos que identifica
        univocamente a la disposicion. Entonces se tiene un diccionario id -> disposicion.

    Las resoluciones por map-reduce son prácticamente iguales a las versiones hechas en mongodb
        ya que la entrada de la función map es practicamente la misma.

Parte 3:
    TODO: primero habría que hacerlo en mongodb jaja.



\newpage
\section{Conclusiones}

Cnclusiones del tp

\end{document}
