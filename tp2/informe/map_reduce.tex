\section{Map-Reduce.}
Intro a MP, explicar como cargar datos y codigo.

\subsection{Disposiciones de tipo resolucion en Abril de 2013.}
IDEA:
map:
si el registro dado es de tipo resolucion y tiene fecha abril del 2013, emitir("resolucion",cant=1)

reduce:
sumar los cant q nos pasan y emitir lo mismo ("resolucion",suma)

\begin{lstlisting}
var map1 = function(){
  var date = this["FechaBOJA"].split('/')
  if(this["Tipo"] == "Resoluciones" && date[1]==4 && date[2]==2013){
    emit(this["Tipo"],1)
  }
}
\end{lstlisting}

\begin{lstlisting}
var reduce1 = function(key,values){
  return Array.sum(values)
}
\end{lstlisting}

luego llamar a la función de map-reduce de la forma:
\begin{lstlisting}
db.disposiciones.mapReduce(map1,reduce1,{out: parte2a})
\end{lstlisting}

Lo que nos devuelve:

\begin{lstlisting}
{ "_id" : "Resoluciones", "value" : 607 }
\end{lstlisting}

\subsection{Disposiciones de cada tipo.}

\subsection{Fecha mas citada.}

\subsection{Mayor cantidad de paginas por cada tipo.}
