\section{Desnormalización.}
Pequeña intro al problema...

\subsection{Empleados que atendieron clientes mayores de edad.}

embebimos los clientes dentro de los empleados, con la fecha de atención, quedando nos asi:

\begin{lstlisting}
Empleado: {
  nroLegajo: int,
  nombre: string,
  clientes: [{DNI, Nombre, Edad, Fecha}],
  ...
}
\end{lstlisting}

Luego, para responder la consulta deseada hay que correr:

\begin{lstlisting}
db.empleados.aggregate(
  [
    { $unwind: "$clientes" },
    { $match: {"clientes.Edad": { $gt: 17 } }  },
    { $group: { _id: "$nombre", nombre:{$first:"$nombre" } } },
    { $project : { _id:0, nombre: 1 } }
  ]
)
\end{lstlisting}
\label{consulta-a}

\subsubsection{Ejemplo}

Para insertar registros en la base corremos:

\begin{lstlisting}
db.empleados.insert( { nroLegajo: 003, nombre: "Ernestino Juanes",
  clientes: [ {DNI: 40528343,  Nombre: "Raul Juan Lopez", Edad: 14,
  Fecha: "14/03/2015"} ] } )
db.empleados.insert( { nroLegajo: 002, nombre: "Juan Paez",
  clientes: [ {DNI: 30154820, Nombre: "Juana Perez", Edad: 23,
  Fecha: "20/04/2015"}, {DNI: 40528753, Nombre: "Raul Lopez", Edad: 15,
  Fecha: "23/04/2015"} ] } )
db.empleados.insert( { nroLegajo: 001, nombre: "Joaquina Paez",
  clientes: [ {DNI: 30154820,  Nombre: "Juan Perez", Edad: 25,
  Fecha: "03/04/2015"} ] } )
\end{lstlisting}

Luego, una vez insertados los registros deseados, corremos la consulta mencionada arriba y nos da:

\begin{lstlisting}
{ "nombre" : "Joaquina Paez" }
{ "nombre" : "Juan Paez" }
\end{lstlisting}

\subsection{Articulos mas vendidos.}
Embebimos los DNIs de clientes que compraron dentro de los artículos y buscamos los máximos.

\begin{lstlisting}
Articulos: {
  CobBarras: int,
  nombre: string,
  compradores: [{DNI}]
}
\end{lstlisting}

Asumimos que hay un único articulo mas vendido:
\begin{lstlisting}
db.articulos.aggregate(
  [
    { $unwind : "$Compradores"},
    {$group : { _id: "$CodBarras", CodBarras:{$first:"$CodBarras"},
      Nombre:{$first:"$Nombre"}, totalVendidos: {$sum: 1} } },
    {$sort: {totalVendidos: -1}},
    {$limit : 1},
    {$project : { _id:0,CodBarras: 1, totalVendidos: 1, Nombre: 1}}
  ]
)
\end{lstlisting}

\subsubsection{Ejemplo}

\begin{lstlisting}
db.articulos.insert( { CodBarras: 7231564345110546, Nombre: "1984",
  Compradores: [22222222] } )

db.articulos.insert( { CodBarras: 0231564105110546,
  Nombre: "El principito", Compradores: [333333333, 22222222] } )
\end{lstlisting}


\subsection{Sectores donde trabaja exactamente 3 empleado.}

\begin{lstlisting}
Sector: {
CodSector: int,
Empleados: [ {NroLegajo, idTarea}]
}

\end{lstlisting}

\begin{lstlisting}
db.sectores.aggregate([{ $unwind : "$Empleados"}, {$group : {_id: "$CodSector",CodSector:{$first:"$CodSector"}, totalEmpleados: {$sum: 1}}},{$project : {_id: 0, CodSector: 1, totalEmpleados:1}},{$match : {totalEmpleados :3 }} ])
\end{lstlisting}

\subsubsection{Ejemplo}
\begin{lstlisting}


db.sectores.insert({CodSector: 1, Empleados: [{NroLegajo: 001, idTarea: 02},{NroLegajo: 002, idTarea: 03},{NroLegajo: 003, idTarea: 01}]})

db.sectores.insert({CodSector: 2, Empleados: [{NroLegajo: 001, idTarea: 01},{NroLegajo: 002, idTarea: 07},{NroLegajo: 003, idTarea: 01},{NroLegajo: 001, idTarea: 02},{NroLegajo: 008, idTarea: 12}]})

db.sectores.insert({CodSector: 4, Empleados: [{NroLegajo: 007, idTarea: 02},{NroLegajo: 003, idTarea: 11}]})


\end{lstlisting}

\subsection{Empleado que trabaja en más sectores..}

\begin{lstlisting}
Empleado: {
nroLegajo: int,
nombre: string,
clientes: [{DNI, Nombre, Edad}]
trabajos: [{CodSector, idTarea}]
}

\end{lstlisting}


\begin{lstlisting}
db.empleados.aggregate([{ $unwind : "$trabajos"}, {$group : {_id: "$nroLegajo",nroLegajo:{$first:"$nroLegajo"}, totalTrabajos: {$sum: 1}}},{$project : {_id: 0, nroLegajo: 1, totalTrabajos:1}},{$sort : {totalTrabajos: -1}},{$limit : 1} ])
\end{lstlisting}

\subsubsection{Ejemplo}

\begin{lstlisting}

db.empleados.insert( { nroLegajo: 006, nombre: "Ernestino Juanes", clientes: [ {DNI: 40528343, Nombre: "Raul Juan Lopez", Edad: 14} ] , trabajos: [{CodSector: 07, Tarea: 10}]} )

db.empleados.insert( { nroLegajo: 005, nombre: "Juan Paez", clientes: [ {DNI: 30154820, Nombre: "Juana Perez", Edad: 23}, {DNI: 40528753, Nombre: "Raul Lopez", Edad: 15} ], trabajos: [{CodSector: 01, Tarea: 02},{CodSector: 04, Tarea: 03},{CodSector: 07, Tarea: 09}] } )

db.empleados.insert( { nroLegajo: 004, nombre: "Joaquina Paez", clientes: [ {DNI: 30154820, Nombre: "Juan Perez", Edad: 25} ], trabajos: [{CodSector: 09, Tarea: 01},{CodSector: 03, Tarea: 05}] } )

\end{lstlisting}


\subsection{Ranking de los clientes con mayor cantidad de compras.}
Asumo que se refiere a ordenarlos por la cantidad de compras que hizo cada uno, porque en el DER no dice nada de ranking ni votos ni nada.
\begin{lstlisting}

Cliente: {
DNI: int,
nombre: string,
edad: int,
compras: [{CodBarra}]
}
\end{lstlisting}

\begin{lstlisting}
db.clientes.aggregate([{ $unwind: "$compras"}, {$group : {_id: "$DNI",DNI:{$first:"$DNI"},nombre:{$first:"$nombre"},totalCompras: {$sum:1}} }, {$project :{ _id:0, DNI:1, nombre:1, totalCompras:1}}, {$sort : {totalCompras : -1}} ] )
\end{lstlisting}
\subsubsection{Ejemplo}
\begin{lstlisting}

db.clientes.insert({DNI: 32012932, nombre: "Guillermo Rodriguez", edad: 23, compras: [{CodBarra: 321},{CodBarra: 023},{CodBarra: 231},{CodBarra: 123}]})

db.clientes.insert({DNI: 33002654, nombre: "Pedro Juanes", edad: 28, compras: [{CodBarra: 023},{CodBarra: 231}]})

db.clientes.insert({DNI: 38165687, nombre: "Carolina Hernandez", edad: 20, compras: [{CodBarra: 123}]})
\end{lstlisting}


\subsection{Cantidad de compras realizadas por clientes de misma edad.}

db.clientes.aggregate([{ $unwind: "$compras"}, {$group : {_id: "$edad", edad: {$first:"$edad"},totalCompras:{$sum:1}}}, {$project: {_id: 0, edad: 1, totalCompras: 1}}])

\subsubsection{Ejemplo}
\begin{lstlisting}
db.clientes.insert({DNI: 33002654, nombre: "Juan Martinez", edad: 28, compras: [{CodBarra: 007},{CodBarra: 109},{CodBarra: 182}]})
\end{lstlisting}
