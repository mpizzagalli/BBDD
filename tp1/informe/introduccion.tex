\section{Introducción.}

En el presente trabajo práctico nos proponemos modelar un registro de accidentes viales a nivel nacional, apartir de este modelo podremos comprender la situación actual, tener mas control sobre lo que sucede y generar posibles politicas o medidas de seguridad de vial. Todo con el fin de poder disminuir la cantidad elevada de accidentes de transito que tenemos hoy en dia, lo cual es una de las preocupaciones del gobierno.\\

De esta forma proponemos la creación de un Registro Único de Accidentes de Tránsito (RUAT), diseñado como presentamos a medida que desarollamos el trabajo práctico. El RUAT almacenará toda la información que resulte relevante a nuestro problema sobre los siniestros ocurridos, como por ejemplo detalles sobre el lugar donde sucedió, la fecha y hora del suceso, condiciones generales del lugar del suceso, personas y vehiculos involucrados en el suceso, informes y peritajes correspondientes, causas probables del siniestro, entre otras.

Por otro lado, el RUAT tambien tendrá información sobre las autopistas y rutas a nivel nacional, información sobre el parque automotor que se encuentra en el país, información sobre quienes y como regulan las polizas de seguro sobre el parque automotor e información sobre los conductores habilitados para conducir y si tienen antecedentes penales.\\

De esta forma el RUAT presenta, desde nuestro punto de vista, información suficiente para que se pueda generar un plan de seguridad eficaz.