\documentclass[a4paper, 12pt, spanish]{article}

\usepackage[paper=a4paper, left=1.5cm, right=1.5cm, bottom=1.5cm, top=3.5cm]{geometry}
\usepackage[spanish, es-noshorthands]{babel}
\usepackage[utf8x]{inputenc}
\usepackage[none]{hyphenat}
\usepackage[colorlinks,citecolor=black,filecolor=black,linkcolor=black,    urlcolor=black]{hyperref}

% Simbolos matemáticos
\usepackage{amsthm}
\usepackage{amsmath}
\usepackage{amsfonts}
\usepackage{amssymb}
\usepackage{algorithm}
\usepackage[noend]{algpseudocode}
\usepackage{algorithmicx}
\usepackage{listings}

% Descoración y gráficos
\usepackage{caratulaV}
\usepackage{graphicx} 
\usepackage{fancyhdr}
\usepackage{lastpage}
\usepackage{caption}
\usepackage{subcaption}
\usepackage{multirow}
\usepackage{alltt}
\usepackage{tikz}
\usepackage{color}
\usepackage{gnuplottex}
\usepackage{verbatim}
\usepackage{framed}
\usepackage[font=small,labelfont=bf]{caption}
\usepackage[normalem]{ulem} %para subrayar

% Bibliografía
\usepackage{natbib}

% Del enunciado
\usepackage{a4wide}
\usepackage{amsmath}
\usepackage{amsfonts}
\usepackage{graphicx}
%\usepackage[ruled,vlined]{algorithm2e}

\newcommand{\kknn}{k}
\newcommand{\kpca}{\alpha}
\newcommand{\kkfold}{K}

% Acomodo fancyhdr.
\pagestyle{fancy}
\thispagestyle{fancy}
\addtolength{\headheight}{1pt}
\lhead{Métodos Númericos}
\rhead{$1^{\mathrm{er}}$ cuatrimestre de 2015}
\cfoot{\thepage /\pageref*{LastPage}}
\renewcommand{\footrulewidth}{0.4pt}

\floatname{algorithm}{Pseudocódigo}
\algrenewcommand\algorithmicfunction{\textbf{Función}}
\algrenewcommand\algorithmicwhile{\textbf{mientras}}
\algrenewcommand\algorithmicfor{\textbf{para}}
\algrenewcommand\algorithmicforall{\textbf{para cada}}
\algrenewcommand\algorithmicdo{\textbf{hacer:}}
\algrenewcommand\algorithmicif{\textbf{si}}
\algrenewcommand\algorithmicthen{\textbf{entonces:}}
\algrenewcommand\algorithmicelse{\textbf{si no:}}
\algrenewcommand\algorithmicend{\textbf{fin}}
\algrenewcommand\algorithmicreturn{\textbf{devolver}}

\sloppy

\parskip=5pt % 10pt es el tama de fuente

% Pongo en 0 la distancia extra entre itemes.
\let\olditemize\itemize
\def\itemize{\olditemize\itemsep=0pt}


%\materia{Métodos Númericos}
%\grupo{Conformación del grupo}
%\tituloCaratula{Trabajo Práctico N$^\circ$1\\ \vspace{0.5cm} ``No creo que a él le gustará eso''}


\usepackage{tikz}
%\usepackage{tikz-qtree}


\usetikzlibrary{arrows,backgrounds,calc}

\pgfdeclarelayer{background}
\pgfsetlayers{background,main}

\newcommand{\real}{\mathbb{R}}
\newcommand{\nat}{\mathbb{N}}

\newcommand{\revJ}[1]{{\color{red} #1}}

\newcommand{\convexpath}[2]{
[ 
    create hullnodes/.code={
        \global\edef\namelist{#1}
        \foreach [count=\counter] \nodename in \namelist {
            \global\edef\numberofnodes{\counter}
            \node at (\nodename) [draw=none,name=hullnode\counter] {};
        }
        \node at (hullnode\numberofnodes) [name=hullnode0,draw=none] {};
        \pgfmathtruncatemacro\lastnumber{\numberofnodes+1}
        \node at (hullnode1) [name=hullnode\lastnumber,draw=none] {};
    },
    create hullnodes
]
($(hullnode1)!#2!-90:(hullnode0)$)
\foreach [
    evaluate=\currentnode as \previousnode using \currentnode-1,
    evaluate=\currentnode as \nextnode using \currentnode+1
    ] \currentnode in {1,...,\numberofnodes} {
-- ($(hullnode\currentnode)!#2!-90:(hullnode\previousnode)$)
  let \p1 = ($(hullnode\currentnode)!#2!-90:(hullnode\previousnode) - (hullnode\currentnode)$),
    \n1 = {atan2(\x1,\y1)},
    \p2 = ($(hullnode\currentnode)!#2!90:(hullnode\nextnode) - (hullnode\currentnode)$),
    \n2 = {atan2(\x2,\y2)},
    \n{delta} = {-Mod(\n1-\n2,360)}
  in 
    {arc [start angle=\n1, delta angle=\n{delta}, radius=#2]}
}
-- cycle
}

\newcommand{\todo}[1]{
\textbf{\color{red}{\underline{Nota:} #1}}
}

\newcommand\param[3]{\ensuremath{\mathbf{\textbf{#1}}\,#2\!:} \texttt{#3}}

\let\state\State
\let\while\While
\let\endwhile\EndWhile
\let\endif\EndIf
\let\elseif\ElsIf
\let\for\For
\let\endfor\EndFor
\let\function\Function
\let\endfunction\EndFunction


\newcommand{\degree}{\ensuremath{^\circ}}

\usepackage{caratula}
\materia{Bases de datos}
\submateria{Segundo Cuatrimestre de 2015}
\titulo{Trabajo Práctico I}
% \subtitulo{\emph{Reentrega}}
\integrante{Ignacio Truffat}{???/??}{el\_truffa@hotmail.com}
\integrante{Gaston Rocca}{???/??}{gastonrocca@gmail.com}
\integrante{Agustín Godnic}{689/10}{agustingodnic@gmail.com}
\integrante{Matías Pizzagalli}{257/12}{matipizza@gmail.com}

\resumen{
En el presente trabajo práctico ...}
\keywords{BBDD, RUAT, MySQL, SQL, ...}

\begin{document}
\maketitle

\newpage

\tableofcontents

\newpage
\section{Introducción.}

En el presente trabajo práctico nos proponemos modelar un registro de accidentes viales a nivel nacional, apartir de este modelo podremos comprender la situación actual, tener mas control sobre lo que sucede y generar posibles politicas o medidas de seguridad de vial. Todo con el fin de poder disminuir la cantidad elevada de accidentes de transito que tenemos hoy en dia, lo cual es una de las preocupaciones del gobierno.\\

De esta forma proponemos la creación de un Registro Único de Accidentes de Tránsito (RUAT), diseñado como presentamos a medida que desarollamos el trabajo práctico. El RUAT almacenará toda la información que resulte relevante a nuestro problema sobre los siniestros ocurridos, como por ejemplo detalles sobre el lugar donde sucedió, la fecha y hora del suceso, condiciones generales del lugar del suceso, personas y vehiculos involucrados en el suceso, informes y peritajes correspondientes, causas probables del siniestro, entre otras.

Por otro lado, el RUAT tambien tendrá información sobre las autopistas y rutas a nivel nacional, información sobre el parque automotor que se encuentra en el país, información sobre quienes y como regulan las polizas de seguro sobre el parque automotor e información sobre los conductores habilitados para conducir y si tienen antecedentes penales.\\

De esta forma el RUAT presenta, desde nuestro punto de vista, información suficiente para que se pueda generar un plan de seguridad eficaz.
\newpage
\section{Diagrama Entidad Relacion.}

\includegraphics[width=0.90\textheight,height=0.99\textwidth,angle=90]{imagenes/DER.png}

\subsection{Restricciones adicionales.}
Las restricciones adicionales que escapan el Diagrama de Entidad Relacion son:
\begin{itemize}
	\item No puede haber dos instancias de la entidad Autopista con mismo nombre.
	\item No puede haber dos instancias de la entidad Ruta con mismo nombre y territorio.
	\item No puede haber dos instancias de la entidad Calle con mismo nombre, ciudad y provincia.
	\item Dado un siniestro, una persona solo puede relacionarse con este siniestro mediante un único rol (Conductor, Acompañante, Testigo, Peaton, Funcionario).
	\item Dado un siniestro, el conjunto de Vehiculos que se relacionan con este siniestro y con la entidad acompañante esta incluido en el conjunto de Vehiculos que se relacionan con este mismo siniestro y con la entidad Conductor.
\end{itemize}


% FALTAN RESTRICCIONES ADICIONALES

\subsection{Asumpciones sobre el dominio del problema.}
A la hora de plantear nuestra solución del problema, tomamos como verdaderas ciertas supociones, las cuales son:

\begin{itemize}
	\item Todo Conductor tiene Licencia de conducir.
	\item Todo Vehiculo tiene al menos un conductor habilitado.
	\item Todo Conductor tiene al menos un Vehiculo para el cual esta habilitado.
	\item Todo Vehiculo tiene una Poliza de seguro.
	\item Si un funcionario se vio involucrado en un siniestro entonces este no puede realizar ningun tipo de informe sobre el siniestro.
	\item Cuando se habla de Tipo de Lugar o Tipo de Camino, lo tomamos como equivalente y lo modelamos en la entidad Via Publica.
	\item La entidad Tipo de Informe representa analisis, peritaje y estudio.
\end{itemize}
\newpage
\section{Modelo Relacional}
\small

Por cuestiones de legibilidad, se ha dividido el modelo relacional en sectores
    de acuerdo al comportamiento que modela cada uno.

\subsection{Modelado general del siniestro}

\textbf{Siniestro}(
    \uline{idSiniestro},
    fecha,
    hora,
    \dashuline{idLugar},
    \dashuline{idTipoAccidente},
    \dashuline{idDenuncia},
    \dashuline{idInforme},
    \dashuline{idCondicion},
    \dashuline{idTipoColision},
    \dashuline{idCausa}
)\\
\textbf{PK} = \textbf{CK} = \{idSiniestro\}\\
\textbf{FK} = \{idLugar, idTipoAccidente, idDenuncia, idInforme, idCondicion, idTipoColision, idCausa\}\\

\textbf{TipoCausaProbable}(
    \uline{idCausa},
    descripcion
)\\
\textbf{PK} = \textbf{CK} = \{idCausa\}\\

\textbf{TipoDeColision}(
    \uline{idTipoColision},
    descripcion
)\\
\textbf{PK} = \textbf{CK} = \{idTipoColision\}\\

\textbf{TipoDeAccidente}(
    \uline{idTipoAccidente},
    descripcion
)\\
\textbf{PK} = \textbf{CK} = \{idTipoAccidente\}\\


\textbf{Participan}(
    \dashuline{\uline{idSiniestro}},
    \dashuline{\uline{dominio}},
    \dashuline{\uline{DNI}},
    usoCintiron
)\\
\textbf{PK} = \textbf{CK} = \{(idSiniestro, DNI, dominio)\}\\
\textbf{FK} = \{idSiniestro, DNI, dominio\}\\


\subsection{Modelado de las personas involucradas en el siniestro}

\textbf{Siniestro}(
    \uline{idSiniestro},
    fecha,
    hora,
    \dashuline{idLugar},
    \dashuline{idTipoAccidente},
    \dashuline{idDenuncia},
    \dashuline{idInforme},
    \dashuline{idCondicion},
    \dashuline{idTipoColision},
    \dashuline{idCausa}
)\\
\textbf{PK} = \textbf{CK} = \{idSiniestro\}\\
\textbf{FK} = \{idLugar, idTipoAccidente, idDenuncia, idInforme, idCondicion, idTipoColision, idCausa\}\\

\textbf{Testigo}(
    \dashuline{\uline{DNI}}
)\\
\textbf{PK} = \textbf{CK} = \{DNI\}\\
\textbf{FK} = \{DNI\}\\
\textit{El caso es análogo para las entidades Peaton y Acompañante.}

\textbf{Participan}(
    \dashuline{\uline{DNI}},
    \dashuline{\uline{idSiniestro}},
    usoCinturon,
    \dashuline{idVehiculo}
)\\
\textbf{PK} = \textbf{CK} = \{(DNI, idSiniestro)\}\\
\textbf{FK} = \{DNI, idSiniestro, idVehiculo\}\\

\textbf{Participa}(
    \dashuline{\uline{DNI}},
    \dashuline{\uline{idSiniestro}}
)\\
\textbf{PK} = \textbf{CK} = \{(DNI, idSiniestro)\}\\
\textbf{FK} = \{DNI, idSiniestro\}\\
\textit{El caso es análogo para la relacion Atestigua.}

\textbf{Persona}(
    \dashuline{\uline{DNI}},
    nombre,
    apellido
)\\
\textbf{PK} = \textbf{CK} = \{DNI\}\\
\textbf{FK} = \{DNI\}\\

\textbf{Antecedente}(
    \uline{idAntecedente},
    fecha,
    descripcion,
    \dashuline{DNI}
)\\
\textbf{PK} = \textbf{CK} = \{idAntecedente\}\\
\textbf{FK} = \{DNI\}\\

\textbf{Denuncia}(
    \uline{idDenuncia},
    fecha,
    \dashuline{DNI},
    \dashuline{idSiniestro}
)\\
\textbf{PK} = \textbf{CK} = \{idAntecedente\}\\
\textbf{FK} = \{DNI\}\\


\subsection{Modelado de un conductor}

\textbf{Conductor}(
    \dashuline{\uline{DNI}}
)\\
\textbf{PK} = \textbf{CK} = \{DNI\}\\
\textbf{FK} = \{DNI\}\\

\textbf{Licencia}(
    \uline{nroLicencia},
    fechaVencimiento,
    \dashuline{DNI},
    \dashuline{idTipoLicencia}
)\\
\textbf{PK} = \textbf{CK} = \{nroLicencia\}\\
\textbf{FK} = \{DNI, idTipoLicencia\}\\

\textbf{TipoLicencia}(
    \uline{idTipoLicencia},
    codigo,
    descripcion
)\\
\textbf{PK} = \textbf{CK} = \{idTipoLicencia\}\\

\textbf{Infraccion}(
    \uline{idInfraccion},
    fecha,
    \dashuline{DNI},
    \dashuline{idTipoInfraccion}
)\\
\textbf{PK} = \textbf{CK} = \{idInfraccion\}\\
\textbf{FK} = \{DNI, idTipoInfraccion\}\\

\textbf{TipoInfraccion}(
    \uline{idTipoInfraccion},
    descripcion
)\\
\textbf{PK} = \textbf{CK} = \{idTipoInfraccion\}\\

\subsection{Modelado de un vehículo}

\textbf{Vehiculo}(
    \uline{dominio},
    fechaPatentamiento,
    \dashuline{idTipoCategoria},
    \dashuline{idTipoVehiculo}
)\\
\textbf{PK} = \textbf{CK} = \{dominio\}\\
\textbf{FK} = \{idTipoCategoria, idTipoVehiculo\}\\

\textbf{Categoria}(
    \uline{idTipoCategoria},
    descripcion
)\\
\textbf{PK} = \textbf{CK} = \{idTipoCategoria\}\\

\textbf{TipoVehiculo}(
    \uline{idTipoVehiculo},
    descripcion
)\\
\textbf{PK} = \textbf{CK} = \{idTipoVehiculo\}\\

\textbf{Poliza}(
    \uline{nroPoliza},
    fechaDesde,
    fechaHasta,
    \dashuline{dominio},
    \dashuline{idTipoCobertura},
    \dashuline{idTipoCompania}
)\\
\textbf{PK} = \textbf{CK} = \{nroPoliza\}\\
\textbf{FK} = \{dominio, idTipoCobertura, idTipoCompania\}\\

\textbf{TipoCobertura}(
    \uline{idTipoCobertura},
    descripcion
)\\
\textbf{PK} = \textbf{CK} = \{idTipoCobertura\}\\

\textbf{TipoCompania}(
    \uline{idTipoCompania},
    CUIT,
    nombre
)\\
\textbf{PK} = \textbf{CK} = \{idTipoCompania\}\\

\textbf{TieneConductorHabilitado}(
    \uline{\dashuline{DNI}},
    \uline{\dashuline{dominio}}
)\\
\textbf{PK} = \textbf{CK} = \{(DNI, idVehiculo)\}\\
\textbf{FK} = \{DNI, dominio\}



\subsection{Lugar de ocurrencia del siniestro}

\textbf{Lugar}(
    \uline{idLugar},
    coordenadasGPS,
    direccion,
    \dashuline{idViaPublica}
)\\
\textbf{PK} = \textbf{CK} = \{idLugar\}\\
\textbf{FK} = \{idViaPublica\}\\

\textbf{ViaPublica}(
    \uline{idViaPublica},
    coordenadasGPS,
)\\
\textbf{PK} = \textbf{CK} = \{idViaPublica\}\\


\textbf{Ruta}(
    \uline{\dashuline{idLugar}},
    KMs,
    territorio,
    numeroRuta
)\\
\textbf{PK} = \textbf{CK} = \textbf{FK} = \{idLugar\}\\

\textbf{Autopista}(
    \uline{\dashuline{idLugar}},
    nombre,
    KMs
)\\
\textbf{PK} = \textbf{CK} = \textbf{FK} = \{idLugar\}\\


\textbf{Calle}(
    \uline{\dashuline{idLugar}},
    provincia,
    ciudad,
    nombre,
    altura
)\\
\textbf{PK} = \textbf{CK} = \textbf{FK} = \{idLugar\}\\


\subsection{Análisis, estudios, peritajes}

\textbf{Informe}(
    \uline{idInforme},
    contenido,
    \dashuline{idTipoInforme}
)\\
\textbf{PK} = \textbf{CK} = \{idInforme\}\\
\textbf{FK} = \{idTipoInforme\}\\

\textbf{Funcionario}(
    \dashuline{\uline{DNI}},
    cargo,
    \dashuline{idOrganismo}
)\\
\textbf{PK} = \textbf{CK} = \{DNI\}\\
\textbf{FK} = \{DNI, idOrganismo\}\\

\textbf{FueRealizadoPor}(
    \dashuline{\uline{DNI}},
    \dashuline{\uline{idInforme}}
)\\
\textbf{PK} = \textbf{CK} = \{(DNI, idInforme)\}\\
\textbf{FK} = \{DNI, idInforme\}\\

\textbf{Organismo}(
    \uline{idOrganismo},
    nombre
)\\
\textbf{PK} = \textbf{CK} = \{idOrganismo\}\\

\textbf{TipoInforme}(
    \uline{idTipoInforme},
    descripcion
)\\
\textbf{PK} = \textbf{CK} = \{idTipoInforme\}\\


\subsection{Condiciones de clima, pavimento, iluminación, etc}

\textbf{CondicionGeneral}(
    \uline{idCondicion},
    elementoSeguridadPeatonal,
    \dashuline{idTipoCondClimatica},
    \dashuline{idTipoDePavimento},
    \dashuline{idEstadoIluminacion},
    \dashuline{idEstadoVia}
)\\
\textbf{PK} = \textbf{CK} = \{idCondicion\}\\
\textbf{FK} = \{idTipoCondClimatica, idTipoDePavimento, idEstadoIluminacion, idEstadoVia\}\\


\textbf{TipoCondClimatica}(
    \uline{idTipoCondClimatica},
    descripcion
)\\
\textbf{PK} = \textbf{CK} =  \{idTipoCondClimatica\}\\

\textbf{TipoDePavimento}(
    \uline{idTipoDePavimento},
    descripcion
)\\
\textbf{PK} = \textbf{CK} = \{idTipoDePavimento\}\\

\textbf{TipoEstadoIluminacion}(
    \uline{idEstadoIluminacion},
    descripcion
)\\
\textbf{PK} = \textbf{CK} = \{idEstadoIluminacion\}\\

\textbf{TipoEstadoVia}(
    \uline{idEstadoVia},
    descripcion
)\\
\textbf{PK} = \textbf{CK} = \{idEstadoVia\}\\

\newpage
\section{Diseño Fisico}

El diseño fisico de nuestra solución lo implementamos sobre el motor de base de datos \texttt{MySQL}.
Constó de varias etapas, primero un modelado con la
    herramienta MySqlWorkbench, y luego desarrollamos scripts para la creación de tablas
    y para poblar las tablas con datos (situados en los directorios \texttt{bd}.
Para crear las tablas se puede correr \texttt{mysql < create.sql}, y
    para llenarlas con datos, correr \texttt{mysql < llenar.sql}.
Esos datos fueron necesarios para probar las queries desarrolladas.
% Aca van los detalles sobre el diseño fisico mas el codigo correspondiente a las consultas/store procedures/trigger pedidos en el enunciado

\subsection{Consulta por número de licencia.}

\subsubsection{version con sub-query}
\lstinputlisting[language=SQL, firstline=1, lastline=53]{"../bd/Consultas/HistorialLicencia DDL.sql"}

\subsubsection{version con Vista.}
\lstinputlisting[language=SQL, firstline=1, lastline=25]{"../bd/Consultas/HistorialLicenciaConVista.sql"}

\subsubsection{version con Trigger}
Los triggers van cargando la tabla participa, hay un trigger por cada tipo de participante.
\lstinputlisting[language=SQL, firstline=1, lastline=53]{"../bd/Consultas/ConsultaModoTrigger/06_HistorialLicenciaConTriggers.sql"}

\subsection{Consulta por modalidad de accidentes viales.}

\lstinputlisting[language=SQL, firstline=1, lastline=27]{../bd/Consultas/DetalleDeAccidentesPorModalidad.sql}


\newpage
\section{Conclusiones}


Nos encontramos con varias soluciones para resolver el problema. Fueron muy valiosas las discuciones de ideas sobre diseños en el DER. También nos percatamos de la complejidad escondida que tienen los diseños, y como algo que parece simple puede ser muy complejo.
Es muy importante invertir tiempo en el diseño para simplificar las consultas y en general el modelo en la base de datos. Tanto fue así, que con el modelo propuesto en el DER logramos soluciones a las consultas bastantes simples, incluso sin necesidad de Triggers.

Para la resolución de las consultas surgieron distintas opciones. La consulta HistorialLicencia se realizo inicialmente con una subquery. Luego transformamos esta subquery en una Vista. Para finalizar realizamos la consulta generando una nueva tabla de Participantes en un siniestro. Esto generó un profundo aprendizage en el uso de los Triggers.

En la generación de los datos tuvimos algunos inconvenientes para carga tablas con muchas FKs. Muchas de las cuales fueron creadas manualmente y con poca cantidad de registros. También nos hubiera gustado generar casos mas complejos, pero por la cantidad de depenedencias se hacia dificultoso.







\end{document}
