\section{Modelo Relacional}
\small

Por cuestiones de legibilidad, se ha dividido el modelo relacional en sectores
    de acuerdo al comportamiento que modela cada uno.

\subsection{Modelado general del siniestro}

\textbf{Siniestro}(
    \uline{idSiniestro},
    fecha,
    hora,
    \dashuline{idLugar},
    \dashuline{idTipoAccidente},
    \dashuline{idDenuncia},
    \dashuline{idInforme},
    \dashuline{idCondicion},
    \dashuline{idTipoColision},
    \dashuline{idCausa}
)\\
\textbf{PK} = \textbf{CK} = \{idSiniestro\}\\
\textbf{FK} = \{idLugar, idTipoAccidente, idDenuncia, idInforme, idCondicion, idTipoColision, idCausa\}\\

\textbf{TipoCausaProbable}(
    \uline{idCausa},
    descripcion
)\\
\textbf{PK} = \textbf{CK} = \{idCausa\}\\

\textbf{TipoDeColision}(
    \uline{idTipoColision},
    descripcion
)\\
\textbf{PK} = \textbf{CK} = \{idTipoColision\}\\

\textbf{TipoDeAccidente}(
    \uline{idTipoAccidente},
    descripcion
)\\
\textbf{PK} = \textbf{CK} = \{idTipoAccidente\}\\


\textbf{Participan}(
    \dashuline{\uline{idSiniestro}},
    \dashuline{\uline{dominio}},
    \dashuline{\uline{DNI}},
    usoCintiron
)\\
\textbf{PK} = \textbf{CK} = \{(idSiniestro, DNI, dominio)\}\\
\textbf{FK} = \{idSiniestro, DNI, dominio\}\\


\subsection{Modelado de las personas involucradas en el siniestro}

\textbf{Siniestro}(
    \uline{idSiniestro},
    fecha,
    hora,
    \dashuline{idLugar},
    \dashuline{idTipoAccidente},
    \dashuline{idDenuncia},
    \dashuline{idInforme},
    \dashuline{idCondicion},
    \dashuline{idTipoColision},
    \dashuline{idCausa}
)\\
\textbf{PK} = \textbf{CK} = \{idSiniestro\}\\
\textbf{FK} = \{idLugar, idTipoAccidente, idDenuncia, idInforme, idCondicion, idTipoColision, idCausa\}\\

\textbf{Testigo}(
    \dashuline{\uline{DNI}}
)\\
\textbf{PK} = \textbf{CK} = \{DNI\}\\
\textbf{FK} = \{DNI\}\\
\textit{El caso es análogo para las entidades Peaton y Acompañante.}

\textbf{Participan}(
    \dashuline{\uline{DNI}},
    \dashuline{\uline{idSiniestro}},
    usoCinturon,
    \dashuline{dominio}
)\\
\textbf{PK} = \{(DNI, idSiniestro)\}\\
\textbf{CK} = \{(DNI, idSiniestro), (idSiniestro, dominio)\}\\
\textbf{FK} = \{DNI, idSiniestro, dominio\}\\
\textit{Esta relación vincula conductores, siniestros y vehículos.}

\textbf{Participa}(
    \dashuline{\uline{DNI}},
    \dashuline{\uline{idSiniestro}}
)\\
\textbf{PK} = \textbf{CK} = \{(DNI, idSiniestro)\}\\
\textbf{FK} = \{DNI, idSiniestro\}\\
\textit{El caso es análogo para la relacion Atestigua.}

\textbf{Participan}(
    \dashuline{\uline{DNI}},
    \dashuline{\uline{idSiniestro}},
    \dashuline{\uline{dominio}},
    usoCinturon
)\\
\textbf{PK} = \textbf{CK} = \{(DNI, idSiniestro)\}\\
\textbf{FK} = \{DNI, idSiniestro, dominio\}\\
\textit{Esta relación vincula acompañantes, siniestros y vehículos.}

\textbf{Persona}(
    \uline{DNI},
    nombre,
    apellido
)\\
\textbf{PK} = \textbf{CK} = \{DNI\}\\

\textbf{Antecedente}(
    \uline{idAntecedente},
    fecha,
    descripcion,
    \dashuline{DNI}
)\\
\textbf{PK} = \textbf{CK} = \{idAntecedente\}\\
\textbf{FK} = \{DNI\}\\

\textbf{Denuncia}(
    \uline{idDenuncia},
    fecha,
    \dashuline{DNI},
    \dashuline{idSiniestro}
)\\
\textbf{PK} = \textbf{CK} = \{idAntecedente\}\\
\textbf{FK} = \{DNI, idSiniestro\}\\


\subsection{Modelado de un conductor}

\textbf{Conductor}(
    \dashuline{\uline{DNI}}
)\\
\textbf{PK} = \textbf{CK} = \{DNI\}\\
\textbf{FK} = \{DNI\}\\

\textbf{Licencia}(
    \uline{nroLicencia},
    fechaVencimiento,
    \dashuline{DNI},
    \dashuline{idTipoLicencia}
)\\
\textbf{PK} = \textbf{CK} = \{nroLicencia\}\\
\textbf{FK} = \{DNI, idTipoLicencia\}\\

\textbf{TipoLicencia}(
    \uline{idTipoLicencia},
    codigo,
    descripcion
)\\
\textbf{PK} = \textbf{CK} = \{idTipoLicencia\}\\

\textbf{Infraccion}(
    \uline{idInfraccion},
    fecha,
    \dashuline{DNI},
    \dashuline{idTipoInfraccion}
)\\
\textbf{PK} = \textbf{CK} = \{idInfraccion\}\\
\textbf{FK} = \{DNI, idTipoInfraccion\}\\

\textbf{TipoInfraccion}(
    \uline{idTipoInfraccion},
    descripcion
)\\
\textbf{PK} = \textbf{CK} = \{idTipoInfraccion\}\\

\subsection{Modelado de un vehículo}

\textbf{Vehiculo}(
    \uline{dominio},
    fechaPatentamiento,
    \dashuline{idTipoCategoria},
    \dashuline{idTipoVehiculo}
)\\
\textbf{PK} = \textbf{CK} = \{dominio\}\\
\textbf{FK} = \{idTipoCategoria, idTipoVehiculo\}\\

\textbf{Categoria}(
    \uline{idTipoCategoria},
    descripcion
)\\
\textbf{PK} = \textbf{CK} = \{idTipoCategoria\}\\

\textbf{TipoVehiculo}(
    \uline{idTipoVehiculo},
    descripcion
)\\
\textbf{PK} = \textbf{CK} = \{idTipoVehiculo\}\\

\textbf{Poliza}(
    \uline{nroPoliza},
    fechaDesde,
    fechaHasta,
    \dashuline{dominio},
    \dashuline{idTipoCobertura},
    \dashuline{idTipoCompania}
)\\
\textbf{PK} = \textbf{CK} = \{nroPoliza\}\\
\textbf{FK} = \{dominio, idTipoCobertura, idTipoCompania\}\\

\textbf{TipoCobertura}(
    \uline{idTipoCobertura},
    descripcion
)\\
\textbf{PK} = \textbf{CK} = \{idTipoCobertura\}\\

\textbf{TipoCompania}(
    \uline{idTipoCompania},
    CUIT,
    nombre
)\\
\textbf{PK} = \textbf{CK} = \{idTipoCompania\}\\

\textbf{TieneConductorHabilitado}(
    \uline{\dashuline{DNI}},
    \uline{\dashuline{dominio}}
)\\
\textbf{PK} = \textbf{CK} = \{(DNI, dominio)\}\\
\textbf{FK} = \{DNI, dominio\}



\subsection{Lugar de ocurrencia del siniestro}

\textbf{Lugar}(
    \uline{idLugar},
    coordenadasGPS,
    direccion,
    \dashuline{idViaPublica}
)\\
\textbf{PK} = \textbf{CK} = \{idLugar\}\\
\textbf{FK} = \{idViaPublica\}\\

\textbf{ViaPublica}(
    \uline{idViaPublica},
    coordenadasGPS,
)\\
\textbf{PK} = \textbf{CK} = \{idViaPublica\}\\


\textbf{Ruta}(
    \uline{\dashuline{idLugar}},
    KMs,
    territorio,
    numeroRuta
)\\
\textbf{PK} = \textbf{CK} = \textbf{FK} = \{idLugar\}\\

\textbf{Autopista}(
    \uline{\dashuline{idLugar}},
    nombre,
    KMs
)\\
\textbf{PK} = \textbf{CK} = \textbf{FK} = \{idLugar\}\\


\textbf{Calle}(
    \uline{\dashuline{idLugar}},
    provincia,
    ciudad,
    nombre,
    altura
)\\
\textbf{PK} = \textbf{CK} = \textbf{FK} = \{idLugar\}\\


\subsection{Análisis, estudios, peritajes}

\textbf{Informe}(
    \uline{idInforme},
    contenido,
    \dashuline{idTipoInforme}
)\\
\textbf{PK} = \textbf{CK} = \{idInforme\}\\
\textbf{FK} = \{idTipoInforme\}\\

\textbf{Funcionario}(
    \dashuline{\uline{DNI}},
    cargo,
    \dashuline{idOrganismo}
)\\
\textbf{PK} = \textbf{CK} = \{DNI\}\\
\textbf{FK} = \{DNI, idOrganismo\}\\

\textbf{FueRealizadoPor}(
    \dashuline{\uline{DNI}},
    \dashuline{\uline{idInforme}}
)\\
\textbf{PK} = \textbf{CK} = \{(DNI, idInforme)\}\\
\textbf{FK} = \{DNI, idInforme\}\\

\textbf{Organismo}(
    \uline{idOrganismo},
    nombre
)\\
\textbf{PK} = \textbf{CK} = \{idOrganismo\}\\

\textbf{TipoInforme}(
    \uline{idTipoInforme},
    descripcion
)\\
\textbf{PK} = \textbf{CK} = \{idTipoInforme\}\\


\subsection{Condiciones de clima, pavimento, iluminación, etc}

\textbf{CondicionGeneral}(
    \uline{idCondicion},
    elementoSeguridadPeatonal,
    \dashuline{idTipoCondClimatica},
    \dashuline{idTipoDePavimento},
    \dashuline{idEstadoIluminacion},
    \dashuline{idEstadoVia}
)\\
\textbf{PK} = \textbf{CK} = \{idCondicion\}\\
\textbf{FK} = \{idTipoCondClimatica, idTipoDePavimento, idEstadoIluminacion, idEstadoVia\}\\


\textbf{TipoCondClimatica}(
    \uline{idTipoCondClimatica},
    descripcion
)\\
\textbf{PK} = \textbf{CK} =  \{idTipoCondClimatica\}\\

\textbf{TipoDePavimento}(
    \uline{idTipoDePavimento},
    descripcion
)\\
\textbf{PK} = \textbf{CK} = \{idTipoDePavimento\}\\

\textbf{TipoEstadoIluminacion}(
    \uline{idEstadoIluminacion},
    descripcion
)\\
\textbf{PK} = \textbf{CK} = \{idEstadoIluminacion\}\\

\textbf{TipoEstadoVia}(
    \uline{idEstadoVia},
    descripcion
)\\
\textbf{PK} = \textbf{CK} = \{idEstadoVia\}\\
