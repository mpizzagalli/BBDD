\section{Diseño Fisico}

El diseño fisico de nuestra solución lo implementamos sobre el motor de base de datos \texttt{MySQL}.
Constó de varias etapas, primero un modelado con la
    herramienta MySqlWorkbench, y luego desarrollamos scripts para la creación de tablas
    y para poblar las tablas con datos (situados en los directorios \texttt{bd}.
Para crear las tablas se puede correr \texttt{mysql < create.sql}, y
    para llenarlas con datos, correr \texttt{mysql < llenar.sql}.
Esos datos fueron necesarios para probar las queries desarrolladas.
% Aca van los detalles sobre el diseño fisico mas el codigo correspondiente a las consultas/store procedures/trigger pedidos en el enunciado

\subsection{Consulta por número de licencia.}

\subsubsection{version con sub-query}
\lstinputlisting[language=SQL, firstline=1, lastline=53]{"../bd/Consultas/HistorialLicencia DDL.sql"}

\subsubsection{version con Vista.}
\lstinputlisting[language=SQL, firstline=1, lastline=25]{"../bd/Consultas/HistorialLicenciaConVista.sql"}

\subsubsection{version con Trigger}
Los triggers van cargando la tabla participa, hay un trigger por cada tipo de participante.
\lstinputlisting[language=SQL, firstline=1, lastline=53]{"../bd/Consultas/ConsultaModoTrigger/06_HistorialLicenciaConTriggers.sql"}

\subsection{Consulta por modalidad de accidentes viales.}

\lstinputlisting[language=SQL, firstline=1, lastline=27]{../bd/Consultas/DetalleDeAccidentesPorModalidad.sql}

